\section{Практика 6. Структуры данных для хэширования}

\subsection{Практика}

\begin{itemize}

    \item Зафиксируем два взаимнопростых числа $n_1$ и $n_2$. Рассмотрим систему уравнений
            $$
            \left\{
                \begin{array}{l}
                    x = a_1 \;\mod\; n_1; \\
                    x = a_2 \;\mod\; n_2,
                \end{array}        
            \right.
            $$
            где $x$ -- неизвестное. Докажите, что для любой пары $(a_1, a_2)$ у данной системы существует ровно одно решение из отрезка $[1, n_1 \cdot n_2]$.
        
        
    \item Пусть мы умеем умножать числа длины $\alpha$ за $M(\alpha)$. Пусть дано $n$ в десятичном виде. Представить эффективный алгоритм поиска $n$-ого числа Фибоначчи по модулю $p$ за
    \begin{enumerate}
        \item $O(\log^2 n \cdot M(\log p))$
        \item $O(\log n \cdot M(\log p))$.
    \end{enumerate}
    Hint: использовать поиск чисел Фибоначчи через матрицы.

    \item Рассмотрим следующий алгоритм.
    \lstinputlisting{source/gcd.py}
    Докажите, что первый элемент возвращенной пары равен $\texttt{gcd}(x, y)$, а второй -- $\texttt{lcm}(x, y)$.    

  % \item Не забудь про функции
  \item Как быстро возвести число в десятичную степень
  \item Что удивило Дейкстру
  \item Двоичный алгоритм для чисел, полином и Евклидова поля
  \item Решето Эратосфена и более быстрые алгоритмы
  \item Деление и корень через разделяй и властвуй

\end{itemize}

\subsection{Домашнее задание. Теория}
\textbf{Дедлайн: 2 апреля, 23.59}

\begin{enumerate}

  \item 
  
\end{enumerate}

\subsection{Что почитать...}

\begin{itemize}
  
\end{itemize}


\clearpage
