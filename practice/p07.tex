\section{Практика 7. Структуры данных для хэширования}

\subsection{Практика}

\begin{itemize}

    \item Зафиксируем два взаимнопростых числа $n_1$ и $n_2$. Рассмотрим систему уравнений
            $$
            \left\{
                \begin{array}{l}
                    x = a_1 \;\mod\; n_1; \\
                    x = a_2 \;\mod\; n_2,
                \end{array}        
            \right.
            $$
        где $x$ -- неизвестное. Докажите, что для любой пары $(a_1, a_2)$ у данной системы существует ровно одно решение из отрезка $[1, n_1 \cdot n_2]$.
        
        
    \item Пусть мы умеем умножать числа длины $\alpha$ за $M(\alpha)$. Пусть дано $n$ в десятичном виде. Представить эффективный алгоритм поиска $n$-ого числа Фибоначчи по модулю $p$ за
    \begin{enumerate}
        \item $O(\log^2 n \cdot M(\log p))$
        \item $O(\log n \cdot M(\log p))$.
    \end{enumerate}
    Hint: использовать поиск чисел Фибоначчи через матрицы.

    \item Рассмотрим следующий алгоритм.
    \lstinputlisting{source/gcd.py}
    Докажите, что первый элемент возвращенной пары равен $\texttt{gcd}(x, y)$, а второй -- $\texttt{lcm}(x, y)$.    

  % \item Решето Эратосфена и более быстрые алгоритмы
  % \item Деление и корень через разделяй и властвуй

\end{itemize}

\subsection{Домашнее задание. Теория}
\textbf{Дедлайн: 2 апреля, 23.59}

\begin{enumerate}

  \item Двоичный GCD.
  \begin{enumerate}
    \item Докажите, что следующий алгоритм находит GCD двух чисел.
      \lstinputlisting{source/gcd2.py}
    \item Обобщите выше приведенный алгоритм до полиномов в $\mathbb{F}[x]$.
  \end{enumerate}

  \item Число называется свободным от квадратов, если его нельзя представить 
  в виде $x^2 \cdot y$ для $x, y \in \mathbb{Z}$ и $x > 1$. Придумайте алгоритм,
  который определяет свободно ли число $n$ от квадратов за $O(n^{\frac{1}{3}})$.
  Арифметичкские действия выполняются за $O(1)$.

  \item По заданным $n$ и $p$, где $p$~-- простое, найдите значение 
    $n! \mod p$ за $O(n \log n)$. Гарантируется, что $p$ укладывается в 
    машинное слово.

  \item По заданным $n, k$ и $p$, $p$ как в предыдущем пункте, требуется 
    посчитать $\binom{n}{k} \mod p$ за $O(n \log n)$. Осторожно, нюансы.


\end{enumerate}



\clearpage
