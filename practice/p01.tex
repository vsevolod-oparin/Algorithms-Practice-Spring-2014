\section{Практика 1. AVL-деревья}


\subsection{Практика}

\begin{enumerate}
  \item Придумать структуру данных (СД), которая поддерживает множество $S \subseteq \mathbb{N}$ со следующими операциями:
    \begin{itemize}
      \item добавить элемент $x$;
      \item удалить элемент $x$;
      \item найти $i$-ый по порядку элемент;
      \item найти порядок элемента $x$ --
    \end{itemize}  
    за $O(\log n)$.

  \item Придумать СД, которая поддерживает множество $S \subseteq \mathbb{N}$ со следующими операциями:
    \begin{itemize}
      \item добавить элемент $x$;
      \item удалить элемент $x$;
      \item найти сумму всех чисел на отрезке $[l, r]$, лежащих в множестве $S$ --
    \end{itemize}  
    за $O(\log n)$.
    
    \textbf{Примечание.} Реализовать можно любую ассоциативную операцию.    

  \item \href{http://acm.timus.ru/problem.aspx?space=1&num=1028}{Задача Stars с тимуса.}

  \item Найти число инверсий в массиве через BST за $O(n \log n)$.

  \item Придумать СД, которая поддерживает множество точек $S \subseteq \mathbb{N}^2$ со следующими операциями:
    \begin{itemize}
      \item добавить точку $x$;
      \item удалить точку $x$;
      \item найти минимальную площадь прямоугольников, который покрывает все точки из $S$, чьи координаты по оси $Ox$ лежат в отрезке $[l, r]$ --
    \end{itemize}
    за $O(\log n)$.

  \item Доказать, что в BST можно вывести $k$ подряд идущих элементов за $O(k + \log n)$.

\end{enumerate}

\subsection{Домашнее задание}
\textbf{Дедлайн: 19 февраля, 23.59}


\clearpage
