\section{Практика 6. Структуры данных для хэширования}

\subsection{Практика}

\begin{itemize}
  \item Рассказать про $k$-независимые функции через полиномы.
 
  \item Обычно в хэш-таблицах используется одна хэш-функция. Мы заведем две $f : K \rightarrow [m]$
   $g : K \rightarrow [m]$. Будем поддерживать инвариант, что любой элемент $x$ лежит либо в ячейке
    $T[f(x)]$, либо в $T[g(x)]$. Если разместить все ключи не получается, выберем
    новую пару функций и перемешаем таблицу. Возьмем $m = 4n$, где $n$ число ключей,
    которые будут добавлены.

    \begin{enumerate}
      \item Придумайте процедуры вставки, удаления и проверки элемента в 
        таблице. При добавлении можно считать, что всегда есть расположение текущих
        элементов, согласующееся с инвариантом.

      \item Нарисуйте граф на $m$ вершинах. Для каждого элемента, который есть в таблице,
        добавьте ребро $(f(x), g(x))$. Пусть функции $f$ и $g$ являюстя $t$-независимыми.
        Разрешим процедуре вставки сделать не более $t$ шагов. Найдите мат. ожидание
        числа шагов во время вставки.

      \item Дайте оценку сверху на вероятность сделать $t$ шагов, при условии, что ни 
        один элемент не будет перемещен дважды. На графе это соответствует обычному пути.

      \item Пусть при вставке, мы дошли обратно до элемента $x$, переместили его в ячейку
        $g(x)$, но дальше в цикл не попали. Оцените вероятность, что мы сделаем $t$ шагов.
  
      \item Оцените вероятность, что в предыдущем пункте, после перехода в ячейку $g(x)$
        мы попадем в цикл. Используйте счетный аргумент.

      \item Подберите $t$ так, чтобы вероятность перемешивания таблицы была $O(n^{-2})$.
      
      \item Докажите, что при соответствующем $t$, такое хэширование использует $O(n)$ памяти,
        все процедуры работают амортизационно по мат. ожиданию $O(1)$.
        
    \end{enumerate}
  
   \item Кратко про фильтр Блума.
    
\end{itemize}

\subsection{Домашнее задание. Программирование}
\textbf{Дедлайн: 6 апреля, 23.59}

Выслал отдельным файлом.

\subsection{Домашнее задание. Теория}
\textbf{Дедлайн: 26 марта, 23.59}

\begin{enumerate}
  \item Пусть дана хэш-таблица размера $n$ с хэш-функцией 
$h : K \rightarrow [n]$. На вход поступает $n$ ключей. Будем предполагать, 
что хэш-функция отправляет каждый ключ в каждую ячейку независимо с равной 
вероятностью. Коллизии разрешаются с помощью односвязных списков, цепочек.
      
      Посчитаем максимальную длину цепочки.
      \begin{enumerate}
          \item Зафиксируем хэш-значение $x$. Доказать, что вероятность, что 
          $k$ ключей будут иметь хэш $x$ составляет
          $$
              Q_k = \binom{n}{k} \cdot \left( \frac{1}{n} \right)^k \cdot  
              \left( 1 - \frac{1}{n} \right)^{n - k}.
          $$
          \item Пусть $P_k$ -- вероятность максимальной цепочки иметь длину $k$.
             Доказать, что $P_k \leq n \cdot Q_k$.
          \item Вывести из Стрилинга, что $Q_k < (\frac{e}{k})^k$.
          \item Показать, что для некоторого $c > 1$ верно $Q_k \leq 
            \frac{1}{n^3}$ при 
          $k \geq c \cdot \frac{\log n}{\log \log n}$.
          \item Доказать, что мат.ожидание длины цепочки не превосходит
          $$
              \Prbb{}{ M > c \cdot \frac{\log n}{\log \log n} } \cdot n + 
              \Prbb{}{ M \leq c \cdot \frac{\log n}{\log \log n} } \cdot c 
              \cdot \frac{\log n}{\log \log n},
          $$
          где $M$~-- максимальная длина цепочки. Обратите внимание, что $M$~--
          случайная величина.
          
          Вывести оценку сверху $O(\frac{\log n}{\log \log n})$.
      \end{enumerate}
  
\end{enumerate}

\subsection{Что почитать...}

\begin{itemize}

% \item http://habrahabr.ru/post/142589/ -- полиномиальное хэширование, статья на Хабре.
% \item http://www.mii.lt/olympiads_in_informatics/pdf/INFOL119.pdf -- более взрослая статья поляков
\item Локально-чувствительные хэш-функции в \href{http://infolab.stanford.edu/~ullman/mmds/book.pdf}{книжке Ульмана}.
\item \href{http://www.it-c.dk/people/pagh/papers/cuckoo-undergrad.pdf}{Куку-хэширование для студентов}.
\item \href{http://www.eecs.harvard.edu/~kirsch/pubs/bbbf/rsa.pdf}{Про функции к фильтру Блума}.
\item \href{http://trac.astrometry.net/export/23660/trunk/documents/papers/dstn-review/papers/bloom1970.pdf}{Оригинальная статья Блума}.
  
\end{itemize}


\clearpage
