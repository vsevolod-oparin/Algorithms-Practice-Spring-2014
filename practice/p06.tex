\section{Практика 6. Структуры данных для хэширования}

\subsection{Практика}

\begin{itemize}
  \item Рассказать на разборе про $k$-независимые функции через полиномы.
  
  \item Пусть дана хэш-таблица размера $n$ с хэш-функцией $h : K \rightarrow [n]$. На вход поступает $n$ ключей. Будем предполагать, что хэш-функция отправляет каждый ключ в каждую ячейку независимо с равной вероятностью.
      
      Посчитаем длину максимальной по длине целочки.
      \begin{enumerate}
          \item Зафиксируем хэш-значение $x$. Доказать, что вероятность, что $k$ ключей будут иметь хэш $x$ составляет
          $$
              Q_k = \binom{n}{k} \cdot \left( \frac{1}{n} \right)^k \cdot  \left( 1 - \frac{1}{n} \right)^{n - k}.
          $$
          \item Пусть $P_k$ -- вероятность максимальной цепочки иметь длину $k$. Доказать, что $P_k \leq n \cdot Q_k$.
          \item Вывести из Стрилинга, что $Q_k < (\frac{e}{k})^k$.
          \item Показать, что для некоторого $c > 1$ верно $Q_k \leq \frac{1}{n^3}$ при 
          $k \geq c \cdot \frac{\log n}{\log \log n}$.
          \item Доказать, что мат.ожидание длины цепочки не превосходит
          $$
              \Prbb{}{ M > c \cdot \frac{\log n}{\log \log n} } \cdot n + \Prbb{}{ M \leq c \cdot \frac{\log n}{\log \log n} } \cdot c \cdot \frac{\log n}{\log \log n}.
          $$
          
          Вывести оценку сверху $O(\frac{\log n}{\log \log n})$.
      \end{enumerate}
  
  \item В этой задаче мы построим совершенное хэширование. Это такая таблица, 
  которая позволяет по имеющимся $n$ ключам из множества $K$, построить таблицу
  использующую $O(n)$ памяти и отвечающую на каждый запрос за $O(1)$ в худшем 
  случае.
  \begin{enumerate}
    \item Пусть дана СВ $X$, принимающая целые неотрицательные значения. 
      Известно, что $\Exb{}{ X > 0 } \leq \frac{1}{2}$. Докажите, что
      $\Prbb{}{X > 0} \leq \frac{1}{2}$.

    \item Пусть есть множество ключей $K$, и для любого $m$ мы можем подобрать
      универсальное семейство хэш-функций $\H_m = \{ h_i : K \rightarrow [m] \}$.
      Подберите такое $m$, что по мат. ожиданию за константное число попыток
      можно подобрать хэш-функцию, не дающую коллизий.

    \item Пусть $\H_m = \{ h_i : K \rightarrow [m] \}$ универсальное семейство 
      хэш-функций.  В таблице $T$ будем разрешать коллизии через цепочки. 
      Пусть $C_i$ -- длина цепочки в ячейке $i$. Докажите, что 
        $$\Exb{h}{ \sum_i \frac{C_i \cdot (C_i - 1)}{2} } = \frac{n \cdot (n - 1)}{2} \cdot \frac{1}{m}$$.

    \item Пусть теперь каждую цепочку в табоице $T$ мы заменим на хэш-таблицу 
    $T_i$ со своей хэш-функцией $h_i$. Подберите размеры таблиц $T_i$, чтобы 
    в них не было коллизий и размер таблицы $T$, чтобы суммарная память занимаемая
    всеми таблицами была $O(n)$.

    \item Докажите, что по имеющимся $n$ ключам можно построить хэш-таблицу по 
    мат. ожиданию за $O(n)$ времени и $O(n)$ памяти. Покажите, что каждый запрос
    обрабатывается за $O(1)$.
    
    \item Добавьте возможность вставлять новые элементы за амортизационное время и память
    $O(1)$. Используйте идею динамических массивов.

  \end{enumerate}

  \item Пусть есть $\H$ -- $6 \log n$-независимое семейство хэш-функции, и
  две функции $f$ и $g$ из него.


\end{itemize}


\subsection{Домашнее задание. Теория}
\textbf{Дедлайн: 26 марта, 23.59}

\begin{enumerate}

  \item 
\end{enumerate}

\subsection{Что почитать...}

\begin{itemize}

% \item http://habrahabr.ru/post/142589/ -- полиномиальное хэширование, статья на Хабре.
% \item http://www.mii.lt/olympiads_in_informatics/pdf/INFOL119.pdf -- более взрослая статья поляков
\item Про локально-чувствительные хэш-функции подробно написано в \href{http://infolab.stanford.edu/~ullman/mmds/book.pdf}{книжке Ульмана}
  
\end{itemize}


\clearpage
