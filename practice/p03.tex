\section{Практика 3. Splay- и декартовы деревья}

\subsection{Практика}

\begin{enumerate}

  \item Операции на отрезках в splay- и декартовых деревьях.

\end{enumerate}

\subsection{Домашнее задание. Программирование}
\textbf{Дедлайн: 9 марта, 23.59}

\begin{enumerate}

  \item Сдать \href{http://acm.timus.ru/problem.aspx?space=1&num=1439}
{Задачу 1439} с Тимуса, если не сдали.

  \item Сдать \href{http://acm.timus.ru/problem.aspx?space=1&num=1521}
{Задачу 1521}. Можно использовать splay- или декартово дерево. Дополнительный 
  балл за реализацию обоих решений.

  \item* Сдать \href{http://www.e-olimp.com/problems/689}{Задачу Своппер} с 
е-Олимпа.

\end{enumerate}

\subsection{Домашнее задание. Теория}
\textbf{Дедлайн: 5 марта, 23.59}

\begin{enumerate}

  \item (Группа Давыдов) Досдать теорему о статической оптимальности splay-дерева.
  Предыдущее задание, задача 4.

\end{enumerate}

\subsection{Что почитать...}

\begin{itemize}
  
  \item Декартовы деревья на Хабрахабре: 
\href{http://habrahabr.ru/post/101818/}{Часть 1}, 
\href{http://habrahabr.ru/post/102006/}{Часть 2},
\href{http://habrahabr.ru/post/102364/}{Часть 3}.

  \item Еще немного \href{http://habrahabr.ru/post/112394/}{про неявный ключ и 
  Rope} на Хабрахабре.
  
  \item \href{http://habrahabr.ru/post/210296/}{Про splay-деревья} на 
  Хабрахабре.

\end{itemize}


\clearpage
