\section{Практика 12. Суффиксные деревья}

\subsection{Практика}

\begin{itemize}

  \item Дана строка из $n$ символов латинского алфавита. Посчитать число различных 
  подстрок за $O(n)$.

  \item Дана строка из $n$ символов латинского алфавита. Рефреном называется
  вхождение, достигающее максимум по своей длине умножить на число вхождений.
  Найти рефрен за $O(n)$.

  \item Даны две строки длины $n$ и $m$. Найти наибольшую общую подстроку
  за $O(n + m)$.

  \item Дан текст длины $n$. Разрешается сделать препроцессинг за $O(n)$.

  \item Построить суф. массив по суф. дереву.

\end{itemize}

\subsection{Домашнее задание. Теория}
\textbf{Дедлайн: 21 мая, 23.59}

\begin{enumerate}

  \item 

\end{enumerate}



\clearpage
