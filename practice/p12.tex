\section{Практика 12. Строковые алгоритмы}

\subsection{Практика}

\begin{itemize}

  \item Дана строка из $n$ символов латинского алфавита. Посчитать число различных 
  подстрок за $O(n)$.

  \item Даны две строки длины $n$ и $m$. Найти наибольшую общую подстроку
  за $O(n + m)$.

  \item Дан текст длины $n$. Разрешается сделать препроцессинг за $O(n)$.
  Требуется отвечать на запросы вида найти первое вхождение заданного 
  паттерна длины $m$. Отвечать на каждый запрос за $O(m)$.

\end{itemize}

\subsection{Домашнее задание. Теория}
\textbf{Дедлайн: 21 мая, 23.59}

\begin{enumerate}

  \item Дана строка из $n$ символов латинского алфавита. Рефреном называется
  вхождение, достигающее максимум по своей длине умножить на число вхождений.
  Найти рефрен за $O(n)$.

  \item Даны две строки $s$ и $t$ длины $n$. Разбейте строку $t$ на набор префиксов
  строки $s$ за линейное время или сообщите, что это невозможно.

  \item Есть некотора строка $x$. Призрак Вася сидит на некоторой позиции и смотрит исключительно
  вперед, т.е. видит ее суффикс. Вася подозревает, что в конце строки стоит зеркало и отржает все символы
  строки в обратном направлении. В зеркале Вася не отображается, но он очень хочет узнать как выглядит 
  целиком строка $x$. По заданной строке $s$, которую видит Вася найдите все возможные варианты строки $x$.
  Время линейно.


\end{enumerate}



\clearpage
