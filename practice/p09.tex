\section{Практика 9. Преобразование Фурье}

\subsection{Практика}

\begin{itemize}


  \item Даны две последовательности $p$ и $q$ действительных чисел от $0$ до $1$ длины не более чем $n$. 
  Известно, что $|q| \geq |p|$. Требуется найти такой сдвиг $s$, что
  $$
    \sum_{i = 0}^{|p| - 1} (p_i - q_{i + s})^2
  $$
  было бы минимальным. Решить за $O(n \log n)$.

  Идея: перемножить два полинома: один нормально, другой развернуть.

  \item Пусть есть полином $p(x) = a_0 + a_1 \cdot x^1 + \cdots + a_{n - 1} \cdot x^{n - 1}$.
  Известны все значения полинома в точках $\omega_n^i$, где 
  $\omega_n = e^{\frac{2 \cdot \pi \cdot i}{n}}$. На вход подается новое значение $a_n$.
  Требуется сдвинуть коэффициенты и посчитать значения для $\omega_n^i$ для полинома
  $p'(x) = a_1 + a_2 \cdot x^1 + \cdots + a_n \cdot x^{n - 1}$.

  Идея: делается в лоб.

  \item Даны две строки $p$ и $q$. Длины обеих строк не превышают $n$. 
  Строка $q$ состоит только из символов $a$ и $b$.
  Строка $p$ может содержать $a$, $b$ и ``$?$''. Здесь знак ``$?$'' выступает в качестве
  wildcard (т.е. под него можно подставить как $a$, так $b$). Найдите все вхождения
  строки $p$ в строку $q$ за $O(n \cdot \log n)$.

\end{itemize}

\subsection{Домашнее задание. Теория}
\textbf{Дедлайн: 16 апреля, 23.59}
\begin{enumerate}

  \item Даны два множества $A$ и $B$ целых чисел из отрезка $[0, 10 \cdot n]$.
  Известно, что размеры множеств не превышают $n$. Требуется найти множество
  $$
    C = \{ a + b\;|\;a\in A, b \in B\}
  $$
  за время $O(n \cdot \log n)$.
  
  Hint: производящие ряды.


  \item По заданным комплексным $z_i$ и неотрицательным целым $\alpha_i$ посчитайте
  коэффициенты полинома $\prod_i (x - z_i)^{\alpha_i}$. Пусть $n = \sum \alpha_i$.
  Решите задачу за $O(n \cdot \log n)$.

  Hint: разделяй и властвуй.  


  \item Матрица $A = (a_{i,j})$ размерности $n \times m$ является Теплицевой, 
  если $a_{i,j} = a_{i + 1, j + 1}$ для всех $1 \leq i \leq n - 1$ и $1 \leq j \leq m - 1$.
  \begin{itemize}
    \item Найдите компактное представление матрицы.
    \item Придумайте быстрый алгоритм для умножения Теплицевой матрицы на вектор за $O((n + m) \cdot \log (n + m))$.    
  \end{itemize} 


\end{enumerate}



\clearpage
