\section{Практика 8. RSA}

\subsection{Практика}

\begin{itemize}

  \item RSA. Алиса шифрует сообщение $m$ с открытым ключом $(N, e)$. Внезапно 
$N$ оказалось простым. Дешифруйте сообщение $m^e$ за $O(poly(\log N))$.

  \item Дано число $N = p \cdot q$ и $\phi(N)$. Факторизовать $N$ за 
  $O(poly(\log N))$.

  % \item Решето Эратосфена и более быстрые алгоритмы
  % \item Деление и корень через разделяй и властвуй

\end{itemize}

\subsection{Домашнее задание. Теория}
\textbf{Дедлайн: 9 апреля, 23.59}

\begin{enumerate}

  \item (группа Опарина) Алёна отправила сообщение $m$, зашифрованное через RSA, трем людям. 
  Для каждого человека
  определено свое $N_i = p_i \cdot q_i$, но везде одинаковое $e = 3$
  Найдите сообщение Алёны за $\texttt{poly} \log N$. 

  \item RSA. В распоряжении взломщика появился ''волшебный'' оракул. Для любого 
  открытого ключа $(N, e)$ оракул
  может взломать 1\% из возможных зашифрованных сообщений. Придумайте алгоритм, 
  который взламывает любое сообщение по матожиданию за $O(poly(\log n))$.

  \item RSA. Пусть есть $N$, $e$ и $d$. Пусть $e = 3$. Разложить $N$ на 
  множители.

  \item Даны натуральные числа $n$ и $p$. $n$ -- большое, $p$ -- маленькое и 
  простое. Пусть $n! = \alpha \cdot p^k$. Требуется найти $\alpha \mod p$ и $k$
  за $O(\texttt{poly}(\log n, p))$.




\end{enumerate}



\clearpage
