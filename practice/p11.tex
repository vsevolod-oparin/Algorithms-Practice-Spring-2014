\section{Практика 11. Алгоритмы Рабина-Карпа и Кнута-Морриса-Пратта}

\subsection{Практика}

\begin{itemize}

  \item Дана строка $s$. Найти наидлиннейшую подстроку $p$,
    встречающуюся в $s$ хотя бы два раза (возможно вхождения
    перекрываются). Время работы $O(n \log{n})$, где $n$ --- длина
    строки $s$.

  \item Дана картинка размера $m \times n$ и прямоугольный образец меньшего
    размера. Найти все вхождения образца в картинке за $O(m \cdot n)$.

  \item * Найти все подпалиндромы заданной строки. Время работы $O(n
    \log{n}$, где $n$ --- длина строки.
\end{itemize}

\subsection{Домашнее задание. Теория}
\textbf{Дедлайн: 14 мая, 23.59}

\begin{enumerate}

  \item Дана строка $s$. Разрешается сделать линейную предобработку.
    По заданным индексам $a$, $b$, $c$ и $d$ требуется отвечать на запросы
    сравнить лексикографически подстроки $s[a..b]$ и $s[c..d]$. Время на
    обработку запроса $O(\log n)$.
    
  \item Даны две строки $s$ и $t$ длины $n$. Найдите наибольшую общую подстроку 
    $s$ и $t$ за $О(n \log{n})$.

  \item Дан массив целых чисел длины $n$. Нужно определить, является ли он префикс-
    функцией для какой-нибудь строки. Если да, построить пример такой строки. 
    Время -- $O(n)$.

  \item Дана строка $s$ и образец $p$. Проверить, что образец $p$
    входит в строку $s$. Допускаются вхождения с не более чем одной опечаткой (
    заменой символа).  Время работы $O(|s| + |p|^2)$.

\end{enumerate}



\clearpage
