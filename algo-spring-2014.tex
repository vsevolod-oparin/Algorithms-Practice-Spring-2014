\documentclass{article}
\usepackage[utf8]{inputenc}
\usepackage{fullpage}
\usepackage[russian]{babel}
\usepackage{amssymb}
\usepackage{listings}
\usepackage{xcolor}
\usepackage{hyperref}

\usepackage{amsthm,amsmath,amsfonts,amssymb}
\usepackage{tikz}
\usepackage{gensymb}

\newcommand{\Prb}[1]{\underset{#1}{\textbf{Pr}}}
\newcommand{\Ex}[1]{\underset{#1}{\textbf{E}}}
\newcommand{\Prbb}[2]{\underset{#1}{\textbf{Pr}}\left[ \; #2 \;\right]}
\newcommand{\Exb}[2]{\underset{#1}{\textbf{E}}\left[ \; #2 \;\right]}
\newcommand{\F}{\mathbb{F}}
\newcommand{\Z}{\mathbb{Z}}
\renewcommand{\H}{\mathcal{H}}

\lstset{
    language=Python,  
    keywordstyle=\bf,
    commentstyle=\color{gray},
    commentstyle=\color{gray!80!white},
    basicstyle=\ttfamily
}
\hypersetup{
    colorlinks=true,
    urlcolor=blue
}
\newcommand{\hint}[1]{
%\rotatebox{90}{Подсказка: #1}
}
  
\title{Практика по алгоритмам}
\author{Алексей Давыдов, Всеволод Опарин \footnote{Составители сборника не являются авторами самих задач. Авторы не указаны в учебных целях.}}
\date{Весна, 2014}

\begin{document}

\maketitle

\section{Практика 1. AVL-деревья}

\subsection{Практика}

\begin{enumerate}
  \item Придумать структуру данных (СД), которая поддерживает множество $S \subseteq \mathbb{N}$ со следующими операциями:
    \begin{itemize}
      \item добавить элемент $x$;
      \item удалить элемент $x$;
      \item найти $i$-ый по порядку элемент;
      \item найти порядок элемента $x$ --
    \end{itemize}  
    за $O(\log n)$.

  \item Придумать СД, которая поддерживает множество $S \subseteq \mathbb{N}$ со следующими операциями:
    \begin{itemize}
      \item добавить элемент $x$;
      \item удалить элемент $x$;
      \item найти сумму всех чисел на отрезке $[l, r]$, лежащих в множестве $S$ --
    \end{itemize}  
    за $O(\log n)$.
    
   \textbf{Примечание.} Можно также реализовать любую ассоциативную операцию.    


  \item Придумать СД, которая поддерживает множество точек $S \subseteq \mathbb{N}^2$ со следующими операциями:
    \begin{itemize}
      \item добавить точку $x$;
      \item удалить точку $x$;
      \item найти минимальную площадь прямоугольников, который покрывает все точки из $S$, чьи координаты по оси $Ox$ лежат в отрезке $[l, r]$ --
    \end{itemize}
    за $O(\log n)$.

  \item Доказать, что в BST можно вывести $k$ подряд идущих элементов за $O(k + \log n)$.

\end{enumerate}

\subsection{Домашнее задание}
\textbf{Дедлайн: 19 февраля, 23.59}

\begin{enumerate}

  \item Сдать в систему acm.timus.ru задачу  \href{http://acm.timus.ru/problem.aspx?space=1&num=1028}{Stars} на AVL-деревьях.

  \item Доказать, что в AVL-дереве можно вывести $k$ подряд идущих элементов за $O(k + \log n)$.

  \item (Давыдов) Найти число инверсий в массиве через BST за $O(n \log n)$.

  \item (Опарин) В качестве доп. информации в узлах AVL-дерева хранится высота вершины. Размер этого числа составляет $O(\log n)$. Предложите реализацию AVL-дерева, в каждом узле которого хранится $O(1)$ бит доп. информации.

  \item (*) Задано множество точек на плоскости $S \in \mathbb{Z}^2$. Определим уровень точки, как в задаче Stars с тимуса.
  Требуется отвечать на запросы, определить уровень точки $x$ относительно множества $S$ за $O(\log n)$. Сама точка в множество \textbf{не} добавляется.

\end{enumerate}

\subsection{Что почитать...}

\begin{itemize}
  
  \item Для AVL-, 2-4- и красно-черных деревьев существует интересное обобщение ranked trees. Прочитать про них можно в книге Роберта Тарьяна: ``Data Structures and Network Algoritms'', а также в списке литературы той же книги.

\end{itemize}

\clearpage

\section{Практика 2. Splay-деревья}

\subsection{Практика}

\begin{enumerate}

  \item Рассмотрим сортировку через \texttt{splay}-деревья. Докажите, что на 
уже отсортированном массиве она работает за $O(n)$.
  
  \item Агрегация на splay-деревьях. Для фиксированной ассоциативной функции
требуется посчитать ее значение на наборе последовательных элементов.

  \item Дан массив из $n$ чисел от $1$ до $n$. Изначально все числа расположены
в порядке возрастания. Требуется уметь обрабатывать два типа запроса:
  \begin{itemize}
    \item Вернуть элемент, стоящий на $i$-ой позиции.
    \item Переместить под отрезок из элементов с $l$-ой позиции по $r$-ую в 
       начало массива.
  \end{itemize}
  Требуется обработать $m$ запросов за время $O(m \log n)$.

  \item Рассказ про парные кучи. Как устроена. Функция потенциалов. Переход
к дереву.

\end{enumerate}

\subsection{Домашнее задание}
\textbf{Дедлайн: 19 февраля, 23.59}

\begin{enumerate}

  \item Сдать \href{http://acm.timus.ru/problem.aspx?space=1&num=1439}
{Задачу 1439} с Тимуса.

  \item Решите \textbf{теоретически}  \href{http://www.e-olimp.com/problems/689}
{Задачу Своппер} с е-Олимпа за $O((n + m) \log n)$. Дополнительный балл за сдачу
в систему.

  \item Докажите, что в Pairing Heap все операции работают амортизационно за 
$O(\log n)$.

Hint: Используйте преобразование кучи по принципу левый ребенок -- правый сосед. Также
используйте потенциал для splay-деревьев.

  \item Пусть \texttt{splay}-дерево поддерживает множество $S$. Каждый элемент 
$x_i \in S$ был запрошен $p_i * m$ раз, где $m$ -- общее число запросов. 
Гарантируется, что $p_i > 0$. Докажите, что \texttt{splay}-дерево обрабатывает
все запросы за время $O(m \cdot \left[ 1 + \sum_{i} p_i \cdot \log \frac{1}{p_i} 
\right])$.

Hint: Подберите подходящие ранги и воспользуйтесь анализом splay-деревьев.

\end{enumerate}

\subsection{Что почитать...}

\begin{itemize}
  
  \item Оригинальная статья Слейтора и Тарьяна ``Self-Adjusting Binary Search Trees''.
  Там есть описание splay-деревьев; несколько теорем о производительности, включая
  статическую оптимальность; два применения: link-cut trees и splay-бор.
  
  \item Статья Тарьяна и компании про спариваемые кучи: Fredman, Sedgewick, 
  Sleator, Tarjan ``The pairing heap: a new form of self-adjusting heap''.

\end{itemize}

\clearpage

\section{Практика 3. Splay- и декартовы деревья}

\subsection{Практика}

\begin{enumerate}

  \item Операции на отрезках в splay- и декартовых деревьях.

\end{enumerate}

\subsection{Домашнее задание. Программирование}
\textbf{Дедлайн: 9 марта, 23.59}

\begin{enumerate}

  \item Сдать \href{http://acm.timus.ru/problem.aspx?space=1&num=1439}
{Задачу 1439} с Тимуса, если не сдали.

  \item Сдать \href{http://acm.timus.ru/problem.aspx?space=1&num=1521}
{Задачу 1521}. Можно использовать splay- или декартово дерево. Дополнительный 
  балл за реализацию обоих решений.

  \item* Сдать \href{http://www.e-olimp.com/problems/689}{Задачу Своппер} с 
е-Олимпа.

\end{enumerate}

\subsection{Домашнее задание. Теория}
\textbf{Дедлайн: 5 марта, 23.59}

\begin{enumerate}

  \item (Группа Давыдов) Досдать теорему о статической оптимальности splay-дерева.
  Предыдущее задание, задача 4.

\end{enumerate}

\subsection{Что почитать...}

\begin{itemize}
  
  \item Декартовы деревья на Хабрахабре: 
\href{http://habrahabr.ru/post/101818/}{Часть 1}, 
\href{http://habrahabr.ru/post/102006/}{Часть 2},
\href{http://habrahabr.ru/post/102364/}{Часть 3}.

  \item Еще немного \href{http://habrahabr.ru/post/112394/}{про неявный ключ и 
  Rope} на Хабрахабре.
  
  \item \href{http://habrahabr.ru/post/210296/}{Про splay-деревья} на 
  Хабрахабре.

\end{itemize}


\clearpage

\section{Практика 4. RMQ и LCA}

\subsection{Практика}

\begin{enumerate}

  \item Дан массив \texttt{a} длины $n$. Программе приходят запросы двух типов.
    \begin{itemize}
      \item Изменить значение в ячейке \texttt{a[i]}.
      \item Посчитать, сколько раз достигается локальный минимум на подотрезке \texttt{[l, r)}.
    \end{itemize} 
    \textbf{Пример}. На в последовательности $1, 2, 3, 1, 2$ минимум достигается два раза.
    Требуется реализовать алгоритм, обрабатывающий все запросы за $O(\log n)$.

  \item Дано дерево $T$ из $n$ вершин. Расстоянием для вершин $u$ и $v$ дерева $T$
    назовем число ребер в пути от $u$ до $v$; обозначим через $\texttt{dist}(u, v)$.
    Требуется уметь за $O(\log n)$ отвечать на запросы "Какого расстояние $\texttt{dist}(u,v)$ 
    для заданных $u$ и $v$?"
    \hint{использовать LCA.}

  \item Дано дерево $T$ из $n$ вершин. Разрешается сделать предобработку дерева.
  Требуется уметь за $O(1)$ отвечать запросы следующего типа.
  \begin{itemize}
    \item По заданным вершинам $u$ и $v$ определить, является ли эта пара предком-потомком.
    Если да, вернуть предка.
  \end{itemize}   

 

\end{enumerate}


\subsection{Домашнее задание. Программирование}
\textbf{Дедлайн: 23 марта, 23.59}

\begin{enumerate}

  \item Сдать \href{http://contest.yandex.com/contest/Contest.html?contestId=484}{яндекс-контест}.

\end{enumerate}

\subsection{Домашнее задание. Теория}
\textbf{Дедлайн: 12 марта, 23.59}

\begin{enumerate}

  \item Дан массив \texttt{a} длины $n$. Ячейки массива можно покрасить в $n^c$ цветов
  для произвольной константы $c$. Программе приходят запросы трех типов.
  \begin{itemize}
    \item Покрасить отрезок $\texttt{[l, r)}$ за $O(\log n)$ в цвет $x$.
    \item Вернуть цвет ячейки $\texttt{a[i]}$ за $O(\log n)$.
    \item Вернуть покраску всего массива за $O(n)$.
  \end{itemize}

  \item Дан массив $\texttt{a}$ длины $n$. Каждая ячейка хранит значение от $0$ до $n^c$
    для произвольной константы $c$.
    Программе приходят запросы двух типов.
    \begin{itemize}
      \item Поменять значение $\texttt{a[i]}$ ячейки.
      \item Для ячейки $\texttt{a[i]}$ найти первый меньший $\texttt{a[i]}$ элемент справа (такой, что $j > i$ и $\texttt{a[j]} < \texttt{a[i]}$).
    \end{itemize}
    Все запросы требуется обработать за $O(\log n)$.	

  \item Рассмотрим бинарную скошенную систему исчисления. На каждой позиции в 
  скошенной записи числа может стоять цифра 0, 1 или 2. Число 
  $a_k a_{k-1} \cdots a_2 a_1$ в скошенной системе переводится
  в десятичную по формуле $\sum_{i = 1}^k a_i \cdot (2^i - 1)$.

  В скошенной системе исчисления есть два ограничения: цифра 2 может 
  встречаться в записи не более одного раза; все цифры следующих меньших 
  разрядов равны нулю. Докажите, что каждое неотрицательное целое число имеет 
  ровно одну возможную запись в скошенной системе исчисления.

  Пример первых чисел: $0,\,1,\,2,\,10,\,11,\,12,\,20,\,100,\,101\dots$
  
  \item (*) Дано дерево из одной вершины. Требуется уметь отвечать на следующий
  набор за $O(\log n)$:
  \begin{itemize} 
    \item Подвесить новую вершину $u$ к вершине дерева $v$ и вернуть диаметр 
  дерева.
  \end{itemize}
  Диаметр дерева -- это длина самого длинного пути в дереве.

\end{enumerate}

\subsection{Что почитать...}

\begin{itemize}
  
  \item Статья Тарьяна про \href{http://www.lb.cs.cmu.edu/afs/cs.cmu.edu/user/sleator/www/papers/pairing-heaps.pdf}{спариваемые кучи}.

  \item Более сильные оценки на время работы парных куч можно прочитать \href{http://web.eecs.umich.edu/~pettie/papers/focs05.pdf}{здесь}.

  \item Сливаемые кучи можно использовать, например, в алгоритме Черитона-Тарьяна для поиска MST. Прочитать про него можно в книжке Тарьяна, на которую я ссылался раньше.

  \item Еще один метод поиска LCA производится через технику \href{http://neerc.ifmo.ru/wiki/index.php?title=%D0%9C%D0%B5%D1%82%D0%BE%D0%B4_%D0%B4%D0%B2%D0%BE%D0%B8%D1%87%D0%BD%D0%BE%D0%B3%D0%BE_%D0%BF%D0%BE%D0%B4%D1%8A%D0%B5%D0%BC%D0%B0}{двоичного подъема}.

\end{itemize}


\clearpage

\section{Практика 5. Хэширование}

\subsection{Определения}

Ниже есть несколько стандартных определений для семейства хэш-функций. $U(\H)$ обозначает равномерное распределение на семействе хэш-функций $\H$. Обратите внимание, что вероятности считаются по случайно взятой хэш-функции.

\begin{itemize}
  \item  Семейство хэш-функций $\H = \{ h_i : X \rightarrow Y \}_i$ называется универсальным, если 
для любых $x_1 \neq x_2 \in X$ $$\Prb{h \leftarrow U(\H)}\left[ \; h(x_1) = h(x_2) \; \right] = \frac{1}{|Y|}$$.

  \item  Семейство хэш-функций $\H = \{ h_i : X \rightarrow Y \}_i$ называется 2-независимым, если 
для любых $x_1 \neq x_2 \in X, y_1, y_2 \in Y$ $$\Prb{h \leftarrow U(\H)}\left[ \; h(x_1) = y_1 \;\wedge\;h(x_2) = y_2 \; \right] = \frac{1}{|Y|^2}$$.

  \item  Семейство хэш-функций $\H = \{ h_i : X \rightarrow Y \}_i$ называется k-независимым, если 
для любых различных $x_1, x_2, \cdots, x_k \in X$, для любых, возможно совпадающих $y_1, y_2, \cdots, y_k \in Y$ $$\Prb{h \leftarrow U(\H)}\left[ \; \bigwedge_{i=1}^k h(x_i) = y_i \; \right] = \frac{1}{|Y|^k}$$.
\end{itemize}

\subsection{Домашнее задание. Теория}
\textbf{Дедлайн: 19 марта, 23.59}

\begin{enumerate}

  \item Докажите, что из любое 2-независимое семейство хэш-функций является универсальным. Приведите пример семейства, которое является универсальным, но не является 2-независимым.

  \item Докажите, что из любое $k+1$-независимое семейство хэш-функций является $k$-независимым.

  \item Построим семейство хэш-функций $h_{a, b} : \Z_p \rightarrow \Z_p$, где $p$ -- простое.
  Определим $h_{a, b} := a \cdot x + b \text{ mod } p$. Определим семейство 
  $$\H = \{h_{a, b}\;|\; a, b \in \Z_p \}.$$
  Покажите, что $\H$ -- 2-независимое.

\end{enumerate}

\subsection{Что почитать...}

\begin{itemize}
  
  \item Дерево Фенвика можно посмотреть на \href{http://community.topcoder.com/tc?module=Static&d1=tutorials&d2=binaryIndexedTrees}{TopCoder-е}.

  \item \href{http://citeseerx.ist.psu.edu/viewdoc/download?doi=10.1.1.55.5156&rep=rep1&type=pdf}{Статья про скошенный список}.
  
  \item \href{http://www.slideshare.net/ekmett/skewbinary-online-lowest-common-ancestor-search}{Презентация скошенного списка}.

\end{itemize}


\clearpage

\section{Практика 6. Структуры данных для хэширования}

\subsection{Практика}

\begin{itemize}
  \item Рассказать на разборе про $k$-независимые функции через полиномы.
  
  \item Пусть дана хэш-таблица размера $n$ с хэш-функцией $h : K \rightarrow [n]$. На вход поступает $n$ ключей. Будем предполагать, что хэш-функция отправляет каждый ключ в каждую ячейку независимо с равной вероятностью.
      
      Посчитаем длину максимальной по длине целочки.
      \begin{enumerate}
          \item Зафиксируем хэш-значение $x$. Доказать, что вероятность, что $k$ ключей будут иметь хэш $x$ составляет
          $$
              Q_k = \binom{n}{k} \cdot \left( \frac{1}{n} \right)^k \cdot  \left( 1 - \frac{1}{n} \right)^{n - k}.
          $$
          \item Пусть $P_k$ -- вероятность максимальной цепочки иметь длину $k$. Доказать, что $P_k \leq n \cdot Q_k$.
          \item Вывести из Стрилинга, что $Q_k < (\frac{e}{k})^k$.
          \item Показать, что для некоторого $c > 1$ верно $Q_k \leq \frac{1}{n^3}$ при 
          $k \geq c \cdot \frac{\log n}{\log \log n}$.
          \item Доказать, что мат.ожидание длины цепочки не превосходит
          $$
              \Prbb{}{ M > c \cdot \frac{\log n}{\log \log n} } \cdot n + \Prbb{}{ M \leq c \cdot \frac{\log n}{\log \log n} } \cdot c \cdot \frac{\log n}{\log \log n}.
          $$
          
          Вывести оценку сверху $O(\frac{\log n}{\log \log n})$.
      \end{enumerate}
  
  \item В этой задаче мы построим совершенное хэширование. Это такая таблица, 
  которая позволяет по имеющимся $n$ ключам из множества $K$, построить таблицу
  использующую $O(n)$ памяти и отвечающую на каждый запрос за $O(1)$ в худшем 
  случае.
  \begin{enumerate}
    \item Пусть дана СВ $X$, принимающая целые неотрицательные значения. 
      Известно, что $\Exb{}{ X > 0 } \leq \frac{1}{2}$. Докажите, что
      $\Prbb{}{X > 0} \leq \frac{1}{2}$.

    \item Пусть есть множество ключей $K$, и для любого $m$ мы можем подобрать
      универсальное семейство хэш-функций $\H_m = \{ h_i : K \rightarrow [m] \}$.
      Подберите такое $m$, что по мат. ожиданию за константное число попыток
      можно подобрать хэш-функцию, не дающую коллизий.

    \item Пусть $\H_m = \{ h_i : K \rightarrow [m] \}$ универсальное семейство 
      хэш-функций.  В таблице $T$ будем разрешать коллизии через цепочки. 
      Пусть $C_i$ -- длина цепочки в ячейке $i$. Докажите, что 
        $$\Exb{h}{ \sum_i \frac{C_i \cdot (C_i - 1)}{2} } = \frac{n \cdot (n - 1)}{2} \cdot \frac{1}{m}$$.

    \item Пусть теперь каждую цепочку в табоице $T$ мы заменим на хэш-таблицу 
    $T_i$ со своей хэш-функцией $h_i$. Подберите размеры таблиц $T_i$, чтобы 
    в них не было коллизий и размер таблицы $T$, чтобы суммарная память занимаемая
    всеми таблицами была $O(n)$.

    \item Докажите, что по имеющимся $n$ ключам можно построить хэш-таблицу по 
    мат. ожиданию за $O(n)$ времени и $O(n)$ памяти. Покажите, что каждый запрос
    обрабатывается за $O(1)$.
    
    \item Добавьте возможность вставлять новые элементы за амортизационное время и память
    $O(1)$. Используйте идею динамических массивов.

  \end{enumerate}

  \item Обычно в хэш-таблицах используется одна хэш-функция. Мы заведем две $f : K \rightarrow [m]$
   $g : K \rightarrow [m]$. Будем поддерживать инвариант, что любой элемент $x$ лежит либо в ячейке
    $T[f(x)]$, либо в $T[g(x)]$. Если разместить все ключи не получается, выберем
    новую пару функций и перемешаем таблицу. Возьмем $m = 4n$, где $n$ число ключей,
    которые будут добавлены.

    \begin{enumerate}
      \item Придумайте процедуры вставки, удаления и проверки элемента в 
        таблице. При добавлении можно считать, что всегда есть расположение текущих
        элементов, согласующееся с инвариантом.

      \item Нарисуйте граф на $m$ вершинах. Для каждого элемента, который есть в таблице,
        добавьте ребро $(f(x), g(x))$. Пусть функции $f$ и $g$ являюстя $t$-независимыми.
        Разрешим процедуре вставки сделать не более $t$ шагов. Найдите мат. ожидание
        числа шагов во время вставки.

      \item Дайте оценку сверху на вероятность сделать $t$ шагов, при условии, что ни 
        один элемент не будет перемещен дважды. На графе это соответствует обычному пути.

      \item Пусть при вставке, мы дошли обратно до элемента $x$, переместили его в ячейку
        $g(x)$, но дальше в цикл не попали. Оцените вероятность, что мы сделаем $t$ шагов.
  
      \item Оцените вероятность, что в предыдущем пункте, после перехода в ячейку $g(x)$
        мы попадем в путей. Используйте счетный аргумент.

      \item Подберите $t$ так, чтобы вероятность перемешивания таблицы была $O(n^{-2})$.
      
      \item Докажите, что при соответствующем $t$, такое хэширование использует $O(n)$ памяти,
        все процедуры работают амортизационно по мат. ожиданию $O(1)$.
        
    \end{enumerate}
  \item Пусть у нас есть структура данных, которая поддерживает множество $S$.
    Мы добавляем новые элементы в множество и проверяем текущие на 
    принадлежность. Проверка стоит дорого, поэтому мы построим специальный 
    фильтр Блума (англ. Bloom filter) между пользователем и структурой данных. 
    Фильтр позволит отбросить те элементы, которые точно не лежат в $S$. 

    Фильтр Блума состоит из битового массива длины $m$. Изначально массив 
    заполнен нулями. Мы подберем $k$ различных хеш-функций 
    $h_1, …, h_k : K \rightarrow [m]$
    так, чтобы 
    $$\textbf{Pr}_{x \leftarrow D(K)}\left[\bigwedge h_i(x) = y_i\right] 
      = O\left(\frac{1}{m^k}\right).$$
    Т.е. при известном распределении на входе, функции должны распределять $x$
    равномерно на выходе.

    Чтобы добавить элемент $x$ запишем единицу в позиции $h_1(x), …, h_k(x)$.

    Чтобы проверить, есть ли элемент $x$ в $S$, проверим биты в позициях
    $h_1(x), …, h_k(x)$. Если обнаружен хотя бы один ноль, то $x \not\in S$.
    Иначе, $x$ может принадлежать $S$. Ситуацию, когда элемент $x \not\in S$, но 
    все соответствующие ему биты равны единице, назовем ложным срабатыванием.

    \begin{enumerate}
      \item
        Оценить вероятность ложного срабатывания при проверке на принадлежность.
        Считать, что в $S$ лежит $n$ элементов.
      \item
        Найти оптимальное число хэш функций при фиксированных $m$ и $n$.
      \item
        Найти размер фильтра Блума такой, что вероятность ложного
        срабатывания не больше $\frac{1}{2}$.
      \item
        Пусть функции $h_1, …, h_k$ только попарно независимы. Как изменятся 
        ответы на предыдущие вопросы?
      \item
        Пусть известны две попарно независимые функции $f_1, f_2 : K \rightarrow [m]$,
        равномерно распределяющие $x$ при условии, вход идет согласно распределению $Q$.
        
        Постройте три попарно независимые функции $h_1, h_2, h_3: K \rightarrow [m]$,
        равномерно распределяющие $x$ при условии, вход идет согласно распределению $Q$.

        Считайте $m$ большим простым числом.
      \item
        Чтобы объединить два одинаковых фильтра Блума (совпадают размеры и 
        хэш-функции, могут различаться множества), достаточно провести 
        побитовое \textbf{ИЛИ} на массивах. Можно ли также получить 
        пересечение?

        Построим пересечение двух фильтров на множествах $A$ и $B$ с помощью 
        побитового \textbf{И}. Оцените вероятность ложного срабатывания 
        в новом фильтре. Размер пересечения $A$ и $B$ известен.

      \item
        Пусть в фильтре Блума есть $x$ ненулевых бит. В 2007 году Свамидасс и Балди 
        показали, что число элементов в фильтре Блума можно оценить как $n =
        -\frac{m\ln{(1-\frac{x}{m})}}{k}$.
        
        Оцените число элементов в пересечении и обьединении множеств, соответствующих 
        двум заданным фильтрам.

      \item
        Научимся удалять элементы из множества. Нужно обновлять состояние 
        фильтра.
        
        Один из вариантов --- хранить не биты, а число попаданий в данную 
        позицию. 

        Пусть для хранения числа попаданий используется используется $l$-битное 
        слово. При переполнении число попаданий устанавливается в бесконечность.

        Пусть число добавлений $n$. Оцените вероятность того, что хотя бы одно 
        из слов будет установлено в бесконечность. Распределении входов считать
        равномерным.
    \end{enumerate}



\end{itemize}


\subsection{Домашнее задание. Теория}
\textbf{Дедлайн: 26 марта, 23.59}

\begin{enumerate}

  \item 
\end{enumerate}

\subsection{Что почитать...}

\begin{itemize}

% \item http://habrahabr.ru/post/142589/ -- полиномиальное хэширование, статья на Хабре.
% \item http://www.mii.lt/olympiads_in_informatics/pdf/INFOL119.pdf -- более взрослая статья поляков
\item Про локально-чувствительные хэш-функции подробно написано в \href{http://infolab.stanford.edu/~ullman/mmds/book.pdf}{книжке Ульмана}
  
\end{itemize}


\clearpage

\section{Практика 7. Теоретико-числовые алгоритмы}

\subsection{Практика}

\begin{itemize}

    \item Зафиксируем два взаимнопростых числа $n_1$ и $n_2$. Рассмотрим систему уравнений
            $$
            \left\{
                \begin{array}{l}
                    x = a_1 \;\mod\; n_1; \\
                    x = a_2 \;\mod\; n_2,
                \end{array}        
            \right.
            $$
        где $x$ -- неизвестное. Докажите, что для любой пары $(a_1, a_2)$ у данной системы существует ровно одно решение из отрезка $[1, n_1 \cdot n_2]$.
        
        
    \item Пусть мы умеем умножать числа длины $\alpha$ за $M(\alpha)$. Пусть дано $n$ в десятичном виде. Представить эффективный алгоритм поиска $n$-ого числа Фибоначчи по модулю $p$ за
    \begin{enumerate}
        \item $O(\log^2 n \cdot M(\log p))$
        \item $O(\log n \cdot M(\log p))$.
    \end{enumerate}
    Hint: использовать поиск чисел Фибоначчи через матрицы.

    \item Рассмотрим следующий алгоритм.
    \lstinputlisting{source/gcd.py}
    Докажите, что первый элемент возвращенной пары равен $\texttt{gcd}(x, y)$, а второй -- $\texttt{lcm}(x, y)$.    

  % \item Решето Эратосфена и более быстрые алгоритмы
  % \item Деление и корень через разделяй и властвуй

\end{itemize}

\subsection{Домашнее задание. Теория}
\textbf{Дедлайн: 2 апреля, 23.59}

\begin{enumerate}

  \item Двоичный GCD.
  \begin{enumerate}
    \item Докажите, что следующий алгоритм находит GCD двух чисел.
      \lstinputlisting{source/gcd2.py}
    \item Обобщите выше приведенный алгоритм до полиномов в $\mathbb{F}[x]$.
  \end{enumerate}

  \item Число называется свободным от квадратов, если его нельзя представить 
  в виде $x^2 \cdot y$ для $x, y \in \mathbb{Z}$ и $x > 1$. Придумайте алгоритм,
  который определяет свободно ли число $n$ от квадратов за $O(n^{\frac{1}{3}})$.
  Арифметичкские действия выполняются за $O(1)$.

  \item По заданным $n$ и $p$, где $p$~-- простое, найдите значение 
    $n! \mod p$ за $O(n \log n)$. Гарантируется, что $p$ укладывается в 
    машинное слово.

  \item По заданным $n, k$ и $p$, $p$ как в предыдущем пункте, требуется 
    посчитать $\binom{n}{k} \mod p$ за $O(n \log n)$. Осторожно, нюансы.


\end{enumerate}



\clearpage

\section{Практика 8. RSA}

\subsection{Практика}

\begin{itemize}

  \item RSA. Алиса шифрует сообщение $m$ с открытым ключом $(N, e)$. Внезапно 
$N$ оказалось простым. Дешифруйте сообщение $m^e$ за $O(poly(\log N))$.

  \item Дано число $N = p \cdot q$ и $\phi(N)$. Факторизовать $N$ за 
  $O(poly(\log N))$.

  % \item Решето Эратосфена и более быстрые алгоритмы
  % \item Деление и корень через разделяй и властвуй

\end{itemize}

\subsection{Домашнее задание. Теория}
\textbf{Дедлайн: 9 апреля, 23.59}

\begin{enumerate}

  \item (группа Опарина) Алёна отправила сообщение $m$, зашифрованное через RSA, трем людям. 
  Для каждого человека
  определено свое $N_i = p_i \cdot q_i$, но везде одинаковое $e = 3$
  Найдите сообщение Алёны за $\texttt{poly} \log N$. 

  \item RSA. В распоряжении взломщика появился ''волшебный'' оракул. Для любого 
  открытого ключа $(N, e)$ оракул
  может взломать 1\% из возможных зашифрованных сообщений. Придумайте алгоритм, 
  который взламывает любое сообщение по матожиданию за $O(poly(\log n))$.

  \item RSA. Пусть есть $N$, $e$ и $d$. Пусть $e = 3$. Разложить $N$ на 
  множители.

  \item Даны натуральные числа $n$ и $p$. $n$ -- большое, $p$ -- маленькое и 
  простое. Пусть $n! = \alpha \cdot p^k$. Требуется найти $\alpha \mod p$ и $k$
  за $O(\texttt{poly}(\log n, p))$.




\end{enumerate}



\clearpage

\section{Практика 9. Преобразование Фурье}

\subsection{Практика}

\begin{itemize}


  \item Даны две последовательности $p$ и $q$ действительных чисел от $0$ до $1$ длины не более чем $n$. 
  Известно, что $|q| \geq |p|$. Требуется найти такой сдвиг $s$, что
  $$
    \sum_{i = 0}^{|p| - 1} (p_i - q_{i + s})^2
  $$
  было бы минимальным. Решить за $O(n \log n)$.

  Идея: перемножить два полинома: один нормально, другой развернуть.

  \item Пусть есть полином $p(x) = a_0 + a_1 \cdot x^1 + \cdots + a_{n - 1} \cdot x^{n - 1}$.
  Известны все значения полинома в точках $\omega_n^i$, где 
  $\omega_n = e^{\frac{2 \cdot \pi \cdot i}{n}}$. На вход подается новое значение $a_n$.
  Требуется сдвинуть коэффициенты и посчитать значения для $\omega_n^i$ для полинома
  $p'(x) = a_1 + a_2 \cdot x^1 + \cdots + a_n \cdot x^{n - 1}$.

  Идея: делается в лоб.

  \item Даны две строки $p$ и $q$. Длины обеих строк не превышают $n$. 
  Строка $q$ состоит только из символов $a$ и $b$.
  Строка $p$ может содержать $a$, $b$ и ``$?$''. Здесь знак ``$?$'' выступает в качестве
  wildcard (т.е. под него можно подставить как $a$, так $b$). Найдите все вхождения
  строки $p$ в строку $q$ за $O(n \cdot \log n)$.

\end{itemize}

\subsection{Домашнее задание. Теория}
\textbf{Дедлайн: 16 апреля, 23.59}
\begin{enumerate}

  \item Даны два множества $A$ и $B$ целых чисел из отрезка $[0, 10 \cdot n]$.
  Известно, что размеры множеств не превышают $n$. Требуется найти множество
  $$
    C = \{ a + b\;|\;a\in A, b \in B\}
  $$
  за время $O(n \cdot \log n)$.
  
  Hint: производящие ряды.


  \item По заданным комплексным $z_i$ и неотрицательным целым $\alpha_i$ посчитайте
  коэффициенты полинома $\prod_i (x - z_i)^{\alpha_i}$. Пусть $n = \sum \alpha_i$.
  Решите задачу за $O(n \cdot \log n)$.

  Hint: разделяй и властвуй.  


  \item Матрица $A = (a_{i,j})$ размерности $n \times m$ является Теплицевой, 
  если $a_{i,j} = a_{i + 1, j + 1}$ для всех $1 \leq i \leq n - 1$ и $1 \leq j \leq m - 1$.
  \begin{itemize}
    \item Найдите компактное представление матрицы.
    \item Придумайте быстрый алгоритм для умножения Теплицевой матрицы на вектор за $O((n + m) \cdot \log (n + m))$.    
  \end{itemize} 


\end{enumerate}



\clearpage

\section{Практика 10. Максимальный поток}

\subsection{Практика}

\begin{itemize}

  \item Найти паросочетание в двудольном графе, используя максимальный поток.

  \item Недавно сотрудники отдела купили себе большой набор чайных пакетиков, 
который содержит $a_1$ пакетиков чая сорта номер 1, $a_2$ пакетиков чая сорта 
номер 2, ..., $a_m$ пакетиков чая сорта номер $m$. Теперь они хотят знать, на 
какое максимальное число дней им может хватить купленного набора так, чтобы в 
каждый из дней каждому из сотрудников доставался пакетик чая одного из его 
любимых сортов.

  \item Дан ориентированный граф $G = \langle V, E \rangle$ без циклов с 
одним стоком $t$ и одним истоком $s$. Каждое ребро имеет пропускную способность
$c_e$. Блокирующим потоком назовем поток $f : E \rightarrow \mathbb{R}$ такой, 
что
\begin{enumerate}
  \item для любой вершины, кроме стока и истока, суммарный поток равен нулю;
  \item для каждого ребра $e$ выполняется $f_e \leq c_e$;
  \item на любом пути от истока к стоку найдется насыщенное ребро ($c_e = f_e$).
\end{enumerate}
  Найдите блокирующий поток за $O(VE)$.

\end{itemize}

\subsection{Домашнее задание. Практика}
\textbf{Дедлайн: 4 мая 23.59}

\begin{enumerate}

  \item Сдать \href{http://acm.timus.ru/problem.aspx?space=1&num=1449}{Тимус 1449}.

  \item Сдать \href{http://contest2.yandex.ru/contest/511/enter/}{Яндекс.Контест}. 
Джеллейные кошки сдавать по желанию.

\end{enumerate}

\subsection{Домашнее задание. Теория}
\textbf{Дедлайн: 23 апреля, 23.59}
\begin{enumerate}

  \item В турнире участвуют $n$ команд. Каждая команда играет с каждой ровно
один раз. Каждый матч завершается победой одной из команд. За победу команде
дается одно очко, за поражение -- ноль. Чемпионом становится команда, у которой
количество очков больше всех.

  Прошло несколько матчей, и известны их результаты. Требуется определить, может
ли первая команда стать чемпионом. Задачу свести к потоку. Сеть может иметь
$O(n^2)$ вершин.

  \item Дана прямоугольная сетка $n \times m$. Некоторые узлы сетки выделены. 
Для каждого выделенного узла требуется проложить путь по ребрам сети на границу, 
так чтобы никакие два пути вершинно не пересекались (см. рисунок). 
Задачу свести к потоку. Размер сети должен быть пропорционален $n \times m$.

  \begin{center}
\begin{tikzpicture}[ thick, every node/.style={scale=1},level distance=1.7cm, sibling distance=2cm]

  \draw[step=0.4cm,thin, gray] (0, 0) grid (2.4, 3.2);

  \node[circle, fill=black, inner sep=0.7mm] at (1.2, 1.2) {};
  \draw[very thick] (1.2, 1.2) -- (0.8, 1.2) -- (0.8, 0.8) -- (0.0, 0.8);

  \node[circle, fill=black, inner sep=0.7mm] at (0.8, 1.6) {};
  \draw[very thick] (0.8, 1.6) -- (0.4, 1.6) -- (0.0, 1.6);

  \node[circle, fill=black, inner sep=0.7mm] at (1.6, 1.6) {};
  \draw[very thick] (1.6, 1.6) -- (2.0, 1.6) --  (2.0, 0);

  \node[circle, fill=black, inner sep=0.7mm] at (1.2, 2.4) {};
  \draw[very thick] (1.2, 2.4) -- (1.2, 3.2);

  \node[circle, fill=black, inner sep=0.7mm] at (0.8, 2) {};
  \draw[very thick] (0.8, 2) -- (0.8, 3.2);

  \node[circle, fill=black, inner sep=0.7mm] at (1.2, 1.6) {};
  \draw[very thick] (1.2, 1.6) -- (1.2, 2) -- (2, 2) -- (2, 3.2);



\end{tikzpicture}
\end{center}
    

  \item Дана сеть $G = \langle V, E, c, s, t \rangle$, где $c$ -- пропускные 
способности ребер, $s$ и $t$ -- исток и сток соответственно. Посмотрим на 
следующий алгоритм поиска максимального потока. Пусть начальный поток $f$ равен 
нулю.
\begin{enumerate}
  \item Сформируем остаточную сеть $R_f = \langle V, E_f, c_f \rangle$ на основе 
    текущего потока.
  \item Запустим из вершины $s$ BFS и найдем реберное расстояние 
$\texttt{dist(v)}$ от вершины $s$ до каждой.
  \item Построим дополнительную сеть $F = \langle V, E_F, c_f \rangle$.
  Ребро $(u, v)$ входит в сеть $F$ тогда и только тогда, когда оно входит в текущую
  остаточную сеть $R$ и $\texttt{dist(v)} - \texttt{dist(u)} = 1$.
  \item Найдем блокирующий поток в сети $F$ и добавим его к текущему ответу. 
    Вернемся к пункту 1.
\end{enumerate}
  Докажите, что после каждой итерации поиска блокирующего потока расстояние между
  стоком и истоком увеличивается хотя бы на единицу.


\end{enumerate}



\clearpage


\end{document}
